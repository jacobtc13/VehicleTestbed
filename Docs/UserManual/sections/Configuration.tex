\documentclass[../main.tex]{subfiles}

\title{Configuration User Manual}
\begin{document}

\section{Overview}
The configuration system is designed to allow users of the simulation to define and run their own scenarios.
The system saves these scenarios, and all relevant components, to file in an xml format.
These files contain all the information necessary to recreate a scenario in a human readable format.

\section{Composition}
Scenario configurations rely on a few other configurations.  Namely, agents and communications.
In order to create a scenario configuration that is more than an empty map you will need to create at least one configuration of each of the aforementioned types.

\subsection{Scenario}
Scenario configurations are the overarching configuration for the scenario.
They determine which agent configurations to use and where to spawn them in the map, and what communications configuration to use.
They also determine where to save the output files of the data recording systems.

\subsection{Agent}
Agent configurations determine the properties of an agent.
What type of agent it is (e.g. Jackal), what countermeasures it is equipped with, and what name it should be represented with in the simulation.


\subsection{Communication}
Communication configurations determine the radio propagation properties to use in the simulation.
See the communications documentation for more details.
The configuration determines what propagation model to use by default and what other models to use within ranges of frequencies.


\section{Using Configurations}
Configurations can be stored anywhere in storage, but keep in mind that scenario configurations point to agent and communication configurations by file location so take care when moving files.

\subsection{In the Program}

\subsubsection{Scenario}
Before jumping into the simulation, the program will first ask you to select or create a new scenario configuration to use.
You will need to add a communication configuration to use, and can add agent configurations to use as well.
There is always an option to create a new communication or agent configuration.
You can set which starting points to use and which agent configurations to use at those starting points.
You can use the same agent configuration for multiple starting points, the simulation will append a number to the name of all subsequent agents using the same configuration.
E.g. Agent, Agent2, Agent3, etc.

\subsubsection{Agent}
The program uses the reflection system of the Unreal Engine to find all available options for what class of agent and what countermeasures are available.
The name of the agent can be anything, it is merely a way to identify it within the simulation and in data output files.

\subsubsection{Communication}
The program again uses the reflection system to find all propagation models available for use.
These models have been denoted as 'SNR models' because their purpose is to calculate the signal-to-noise ratio between two points.
You will need to set which model to use, and can choose to add other models to use within a range of frequencies instead of the default model.

\subsection{With XML Files}
Scenarios can be created solely in XML however you will not have Unreal's reflection system available to help select the correct name of classes you wish to use.
The names of blueprint classes in particular will not be what one might expect so it is recommended to use the program to create a configuration using a class to find its name from the subsequently created xml document.

\subsubsection{Scenario}
Scenario configurations have the following structure:
\begin{itemize}
	\item Scenario Tag
	\item Map Tag
	\item Agents Tag
	\begin{itemize}
        \item Agent Tag
    \end{itemize}
	\item Spawns Tag
	\begin{itemize}
        \item Spawn Tag
        \begin{itemize}
            \item Name Tag
            \item Agent Tag
        \end{itemize}
    \end{itemize}
	\item Data Recording Output Tag
	\item Event Recording Output Tag
	\item Communication Tag
\end{itemize}

The program can interpret these major tags in any order, but will not look for any sub tags anywhere other than within their major tag.

\paragraph{Scenario Tag}
The scenario tag is used to identify this configuration as a scenario configuration.
Do not remove or change this tag or the program will fail to load this scenario.

\paragraph{Map Tag}
This tag specifies the name of the map to use, this tag must have a value.

\paragraph{Agents Tag}
The agents tag houses a list of tags denoting the agent configuration files used by this scenario.
It is not necessarily required for a scenario to use agents but the scenario will be empty without them.
An agent must be listed here if it is referenced in the spawns tag.

\paragraph{Agent Tag}
This tag specifies the file name of the agent configuration to use.

\paragraph{Spawns Tag}
The spawns tag houses a list of spawn tags which have the name of the spawn point and the configuration file of the agent to spawn.
Like agents, it is not necessarily required that a scenario have any agents and therefore any spawn points.

\paragraph{Spawns- Spawn Tag}
This tag contains the name and agents tags of a spawn point.

\paragraph{Spawns- Spawn- Name Tag}
This tag specifies the name of the spawn point to start an agent at.
The names are defined in the program so it may be useful to open the program to see what the names are.

\paragraph{Spawns- Spawn- Agent Tag}
This tag specifies the agent configuration file to use for this spawn point.

\paragraph{Data Recording Output Tag}
This tag specifies the output folder of the data recording system.
It is recommended that the folder path does not have an ending slash but the system will handle the case that it does.

\paragraph{Event Recording Output Tag}
This tag specifies the output folder of the event recording system.
Like before it is recommended that this value does not have an ending slash.

\paragraph{Communication Tag}
This tag specifies which communication configuration to use by its file name.
This tag must have a value.

\subsubsection{Agent}
Agent configurations have the following information:
\begin{itemize}
    \item Agent Tag
    \item Class Tag
    \item Name Tag
    \item Possess Tag
    \item Gadgets Tag
    \begin{itemize}
        \item Gadget Tag
        \begin{itemize}
            \item Name Tag
        \end{itemize}
    \end{itemize}
\end{itemize}

\paragraph{Agent Tag}
This tag is used to identify the configuration as an agent configuration.

\paragraph{Class Tag}
This tag specifies the name of the class this agent should be.
It must be a perfect match of the name of an agent class in the program.

\paragraph{Name Tag}
This tag specifies the name of agents created by this configuration.
An agent is not required to have a name but it will reduce the readability of any data recording files.

\paragraph{Possess Tag}
This tag specifies whether agents of this configuration should be controlled by the player when the simulation starts.
If more than one agent of this configuration is created the player will be in control of the last one created.

\paragraph{Gadgets Tag}
This tag specifies what gadgets are attached to the agent.
Gadgets include countermeasures.
This tag contains a list of gadget tags which in turn contain a name tag which denotes the name of the class of the gadget.

\paragraph{Gadgets- Gadget Tag}
This tag describes everything needed to create a gadget in the simulation.

\paragraph{Gadgets- Gadget- Name Tag}
This tag specifies the name of the class of this gadget.
The name must match a gadget class in the program.

\subsubsection{Communication}
Communication configurations have the following information:
\begin{itemize}
    \item Communication Tag
    \item Default Model Tag
    \item Frequency Ranges Tag
    \begin{itemize}
        \item Frequency Range Tag
        \begin{itemize}
            \item SNR Model Tag
            \item Min Frequency Tag
            \item Max Frequency Tag
        \end{itemize}
    \end{itemize}
\end{itemize}

\paragraph{Communication Tag}
This tag is used to identify this configuration as a communication configuration.

\paragraph{Default Model Tag}
This tag specifies the name of the propagation model to use.
The name must match a class in the program.

\paragraph{Frequency Ranges Tag}
This tag hosts a list of frequency range tags.

\paragraph{Frequency Ranges- Frequency Range Tag}
This tag specifies an extra propagation model to use within a range of frequencies.

\paragraph{Frequency Ranges- Frequency Range- SNRModel Tag}
This tag specifies the name of the propagation model to use within this range of frequencies.
The name must match a class within the program.

\paragraph{Frequency Ranges- Frequency Range- Min Frequency Tag}
This tag specifies the minimum value of this range.
Inclusive.

\paragraph{Frequency Ranges- Frequency Range- Max Frequency Tag}
This tag specifies the maximum value of this range.
Inclusive.

\end{document}
